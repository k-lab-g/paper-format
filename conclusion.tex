\chapter{結論}
本研究では、非直交 CSK を用いた IM/DD 方式を提案した。
特に、CSK で用いる非直交系列をシステマチックに構成する方法を提案した。
本研究では、シンチレーションが存在する環境下において、
シンボル誤り率および情報伝送効率を導出した。
また、従来の IM/DD 方式である、SIK 方式および 
M-ary 直交変調方式との比較を行った。その結果、以下のことを明かにした。
%
\begin{itemize}
 \item M-ary 直交変調方式のシンボル誤り率が最も良い。
 \item 提案方式は情報伝送効率を最大化する、連接数の最適値が存在する。
 \item 通信環境にかかわらず提案方式の情報伝送効率が最も良い。
\end{itemize}
%
以上より、非直交 CSK を用いた IM/DD 方式は光無線通信における
情報伝送効率向上法として有効な方式である。

今後の課題として、
提案方式における連接数の最適値に関する検討、
提案方式におけるシンボル誤り率低減法に関する検討、
情報伝送効率のさらなる向上法に関する検討、
情報変調の階層化に関する検討、
多元接続干渉を考慮した性能解析、などが挙げられる。
