\documentclass[11pt,oneside]{jreport} %もしくは,jbook

%%% 共通パッケージの読み込み(Required)
\usepackage{graphicx}  % 図の貼込みを可能にする.
%\usepackage{overcite}  % 引用文献番号を右肩に上げる.
\usepackage{amsthesis} % 知能情報コース修論スタイル
\usepackage{amsmath}

%%% オプションパッケージの読込み(Optional)
%\usepackage{here} % 図,表をその位置に貼る.
\usepackage{fancybox} % ボックスの拡張
\usepackage{txfonts}  % TXフォント使用

%%% 修士論文題目等(Required)
\title{\textbf{非直交符号シフトキーイングを用いた光強度変調方式に関する研究}} % ここに論文題目を書く.
\gakuseki{学籍番号}
\date{平成30年2月8日}           % ここに提出日を書く.
\laboratory{小室研究室} % ここに研究室名を書く.(amsthesis only)
\author{小室 信喜}               % ここに氏名を書く.

%%% 各自の定義(Optional)
% ここに個人の定義を書く.
% \def\baselinestretch{1.2} % 改行幅を一律既定値の20%にする.
%\usepackage{amsmath}

%%% Jpeg ファイルを読み込むための仕様
%\DeclareGraphicsRule{.jpeg}{eps}{.bb}% 
%{`djpeg -pnm #1 | pnmtops} 

%%% PDF ファイルのためのポストスクリプトフォントの使用
%\usepackage[T1]{fontenc}
%\usepackage{times}
%\usepackage{mathptm}



%%% PDF ファイルのための設定
 \usepackage[dvipdfm,bookmarks=true,bookmarksnumbered=true,bookmarkstype=toc]{hyperref}
 \hypersetup{a4paper=true,pdfpagemode=None,pdfstartview=FitH}
 \hypersetup{%
   pdftitle={}
   pdfsubject={},
   pdfauthor={},
   pdfkeywords={}
 }

%%% 目次をどこまで出すか(選択)
%\setcounter{tocdepth}{0} % 章まで
%\setcounter{tocdepth}{1} % 節まで
\setcounter{tocdepth}{2} % 項まで

%%% 図,表のキャプションを日本語とするか欧文とするか(選択)
\makeatletter
%\def\fnum@figure{Fig. \thefigure}
%\def\fnum@table{Table \thetable}
\def\fnum@figure{図\thefigure}
\def\fnum@table{表\thetable}
\makeatletter

\renewcommand{\baselinestretch}{1.0} % % 行間設定 1.0が通常、入らなかったら調整
%図と本文の調整、狭くしたかったら小さい値を設定


\begin{document}

%%% 表紙(Required)
\maketitle



%%% 概要(Required)
% 卒論では不要
%\begin{abstract} % (amsthesis only)
%\vspace{5mm}
%\end{abstract} %(amsthesis only)

%%% 目次(Required)
\tableofcontents

% 図目次
\listoffigures

% 表目次
\listoftables
\newpage
\setcounter{page}{1}%

%%% 本文(Required)
\chapter{序論}
近年、車車間通信や海中通信、ホームネットワークへの応用が期待できる
情報通信方式として、光無線通信が着目されている\cite{ocdma1}-\cite{csk2}。
光無線通信サービスの普及に向けたさまざまな標準化活動、
光無線通信の高度化および実用化に関する研究が行われており、
光強度変調/直接検波 (IM/DD: Intensity Modulation Direct Detection) が実用的な光無線通信方式として検討されている。
光強度変調方式では、
光変復調法\cite{ocdma4}-\cite{scinti}、
光無線通信路のモデル化\cite{ocdma5}、
疑似雑音系列を用いたIM/DD方式の情報変調法\cite{ook}-\cite{sik2}、
などが検討されている。

疑似雑音系列を用いたIM/DD方式における情報変調方式として、
オンオフキーイング (OOK: On-off Keying) 方式\cite{ook}や
シーケンスインバージョン (SIK: Sequence Inversion Keying) 方式\cite{sik1}\cite{sik2}が検討されている。
OOK方式は、送信データ"0"を送信する場合は疑似雑音系列を発生させず、送信データ"1"を
送信する場合は疑似雑音系列を発生させ送信する方式である。
OOK方式は、光強度の減衰が少ない
光ファイバ通信では実用的な方式であるが、光無線通信方式では、
シンチレーションが生じるため\cite{scinti}、
動的に理想的なしきい値を設定することが困難である。
SIK方式は直交する2つの疑似雑音系列を用い、情報に応じて送信符号を切り替える
方式である\cite{sik1}\cite{sik2}。
SIK方式は受信機でのしきい値判定を必要とせず、OOK方式と同じ
情報伝送効率を達成することができるため、光無線通信に適した方式である。
しかし1つの拡散系列あたり1ビットのデータを送信するため、
情報伝送効率の向上が課題である。

情報伝送効率の向上を図る情報変調として、M-ary 直交変調方式が
検討されている\cite{csk1}\cite{csk2}。
M-ary 直交変調方式は、符号シフトキーイング (Code Shift Keying; CSK) の
1種であり、伝送するデータに応じて$M$個の直交系列の中から1つの系列を選択し、
送信する方式である。
M-ary 直交変調方式は、直交系列数$M$を増加することに
よって、系列あたりの情報量を増加させることができる。
しかし、系列数を増加させるには系列長を長く必要があるため、情報伝送効率をまねく
おそれがある。

本研究では、光無線通信における情報伝送効率向上法として、非直交 CSK を
用いた光強度変調方式を提案する。特に、CSK で用いる非直交系列をシステマチックに
構成する方法を提案する。
CSK で用いる系列として非直交系列を用いることによって、
系列長を長くせずに情報変調に用いる系列数を増やすことができるため、
情報伝送効率の向上が期待できる。
シンチレーションが存在する環境下において、
提案方式のシンボル誤り率および情報伝送効率の理論式を導出し、
提案方式の有効性を示す。

\chapter{関連研究}
\section{疑似雑音系列を用いた光強度変調システム}
光無線通信では、光強度変調/直接検波 (IM/DD: Intensity Modulation Direct Detection) が
実用的な方式として検討されている。
IM/DD 方式は、送信側で情報信号に応じて光の強度を変調し、
受信側で受信した信号の光強度を測ることにより、光子のエネルギーを検出し、
情報を復調する方式である。
近年では、光符号多分割多元接続 (OCDMA: Optical Code Division Multiple Access) 
への応用を想定し、疑似雑音系列を用いたIM/DD方式が検討されている\cite{ocdma1}-\cite{csk2}。

\subsection{疑似雑音系列を用いたIM/DD方式の原理}
送信機側では、レーザにより放射された光パルスが光スプリッタにより擬似雑音系列の
重み数$W$に分割される。次に疑似雑音符号の重み位置に対応した光時間遅延線を用いて、
分岐された$W$個の光パルスを時間遅延させ、最後に結合器によって光パルスを集め、
疑似雑音符号に従った長さ$L$ (chip)の光パルス列を送信する。
光無線通信路上では、送信信号は大気を伝搬する際の分子吸収による損失や、
粒子による散乱等の影響を大きく受ける。とくに、大気の屈折率変動によって
光ぱするの強度が変動してしまうシンチレーションという現象は光の強度に
情報をのせ送信する光強度変調方式に大きく影響を与える\cite{scinti}。
また、受信光パルス列が
レンズを通して受信機へ入射される際に、照明光や太陽光などの背景光雑音も
同時に入射してしまう。したがって、受信機では次の信号$r(t)$が入射される。
\begin{align}
r(t) &= \sqrt{P_w(t)} PN(t) X(t) + \sqrt{\frac{P_w(t)}{Me}}(1-PN(t)) X(t) + P_b(t)
\end{align}
ここで、$PN(t)$は疑似雑音系列、$P_w(t)$は疑似雑音系列の重みの位置の受信光パワー、
$\frac{P_w(t)}{Me}$は疑似雑音系列の重み位置以外の受信光パワー、
$X(t)$はシンチレーション、$P_b(t)$は背景光雑音、$Me$は変調消光比である。

受信機では、送信側で使用した系列と同一の系列を使用し、相関検波を行う。
相関検波では、受信信号はチップ間隔$T_c$毎にスイッチングされ、送信機とは逆の
時間遅延特性を持つ光遅延線によって、疑似雑音系列の重み位置の全光パルスが、
系列の最終チップに集められる。最終チップに集められた光パワー$P_{in}$は
系列長間隔$LT_c$におけるシンチレーションの影響を$X$とした場合、
\begin{align}
P_{in} &= W(P_w X + P_b)
\end{align}
で表される。

\subsection{疑似雑音系列を用いた光強度変調システムの情報変調方式}
疑似雑音符号を用いたIM/DD方式における情報変調方式として、
オンオフキーイング (OOK: On-off Keying) 方式\cite{ook}や
シーケンスインバージョン (SIK: Sequence Inversion Keying) 方式\cite{sik1}\cite{sik2}が検討されている。
\subsubsection{OOK方式}
OOK方式は光強度変調において最も単純な情報変調法である。OOK方式は、
送信データ"0"を送信する場合は疑似雑音系列を発生させず、送信データ"1"を
送信する場合は疑似雑音系列を発生させ送信する方式である。
受信機では、しきい値を設定することによってパルスの有無を判定し、
データを復調する。そのため、OOKではビット誤り率特性が最適となる理想的な
しきい値の設定が必要となる。したがって、光強度の減衰が少ない
光ファイバ通信では実用的な方式であるが、光無線通信方式では、
シンチレーションが生じるため、動的に理想的なしきい値を設定することが困難である。

\subsubsection{SIK方式}
SIK方式は直交する2つの疑似雑音系列を用い、情報に応じて送信符号を切り替える
方式である。SIK方式は受信機でのしきい値判定を必要とせず、OOK方式と同じ
情報伝送効率を達成することができる。送信機では、データに応じて
疑似雑音系列を1つ選択し、その疑似雑音系列に従って光パルスを時間拡散することで、
1ビットのデータを送信する。受信機では、送信側と同じ系列を用意し、
相関値が最大となる系列を送信系列と推定し、データを復調する。
したがって、受信機側でしきい値を必要とせず、光無線通信に適した方式である。
しかし1つの拡散系列あたり1ビットのデータを送信するため、
情報伝送効率の向上が課題である。
%

\section{M-ary 直交方式の原理}
M-ary 直交変調方式は符号シフトキーイング (CSK: Code Shift Keying) 方式\cite{csk1}\cite{csk2}の1種
である。CSKは、情報データに応じて拡散系列を変化させ、伝送する方式である。
SIK方式は1つの疑似雑音系列を用いて情報を伝送する方式であるのに対し、
CSK は疑似雑音系列の集合の中から情報データによって1つの系列を選択し、
その選択した系列によってデータを伝送する方式である。
M-ary 直交変調方式は$M$個の直交系列の集合の中から、
情報データによって1つの直交系列を選択し、
その選択した系列によって情報を伝送する方式である。
そのため1つの拡散系列あたりのビット数を増やすことができ、
情報伝送効率の向上が期待できる方式である。
%
\begin{figure}[hpbt]
\begin{center}
  \includegraphics[width=1.0\textwidth]{csk_system.eps}
\caption{M-ary 直交変調方式のシステム構成}
\label{csk_system}
\end{center}
\end{figure}
%
図\ref{csk_system}にM-ary 直交変調方式の
送信機および受信機の構成を示す。送受信機で利用する直交系列の集合を
$\{os_i(t); i=1, 2, \cdots, M\}$で表すものとする。送信側ではまず、
これらの$M$個の直交系列の中から$\log_2 M$ビットの情報によって1つの直交系列を
択する。次に、搬送波を乗算することによって、その系列を送信する。

受信側では送信側で用いる直交系列と同じ系列を用いる。
受信信号をチップ間隔ごとにAvalanche Photo Diode (APD)を用いて光電変換する。
光電変換した電気信号は、
要素系列毎に各直交系列$os_{j}~~(j=1,2, \cdots, M)$との相関を取る。
各直交系列の相関値の絶対値の和を求め、その値が最も大きくなる系列を、
送信された系列の要素系列であると判定し、情報を復調する。

%

\chapter{提案システムの構成}

\section{非直交系列構成法}
図\ref{fig:structure}に非直交系列の構成を示す。
本方式では、直交系列を$M_{con}$個連接することにより、非直交系列を構成する。
直交系列を$M_{con}$個連接することにより構成された系列を
1フレームとする。直交系列 \#$i$から構成される非直交系列のグループを
グループ\#$i$とする。提案方式において構成される系列は、
異なるグループの系列とは直交しているが、
同じグループの系列は非直交である。1つの直交系列から $2^{M_{con}}$個の
非直交系列が構成される。したがって、$M_{os}$個の直交系列からは、
$2^{M_{con}} M_{os}(=M_{non})$個の非直交系列が構成される。また、
1非直交系列当たりの情報ビット数は
$N_{bit} = \log_2 M_{os} + M_{con}$[bit] である。
直交系列を長 $L_{os}$ とすると、フレーム長$L_f$は $M_{con} \times L_{os}$ となる。
%
%
\begin{figure}[hpbt]
\begin{center}
  \includegraphics[width=0.75\textwidth]{sequence.eps}
\caption{非直交系列の構成}
\label{fig:structure}
\end{center}
\end{figure}
%

\section{送受信機}
図\ref{fig:model}に本方式の送受信機の構成を示す。
送信側では、まず、送信データ ($N_{bit}$(bit) の情報)を
$\log_2 M_{os}$ (bit)と$M_{con}$ (bit)に分割する。
$\log_2 M_{os}$ (bit)のデータに基づいて$M_{os}$個の直交系列の中から
1つの系列を選択する。次に、$M_{con}$ (bit)の極性に応じて、選択した直交系列を連接する。
最後に、構成された連接系列を送信する。
%

受信側では送信側で用いる直交系列と同じ系列を用いる。
受信信号をチップ間隔ごとにAvalanche Photo Diode (APD)を用いて光電変換する。光電変換した電気信号は、
要素系列毎に各直交系列$OS_{j}~~(j=1,2, \cdots, M_{os})$との相関を取る。
各直交系列の相関値の絶対値の和を求め、その値が最も大きくなる系列を、
送信された系列の要素系列であると判断する。また、相関器出力の正負から
連接系列の極性を判定し、データを復調する。
\begin{figure}[hpbt]
\begin{center}
  \includegraphics[width=1.0\textwidth]{system.eps}
\caption{提案方式のシステムモデル}
\label{fig:model}
\end{center}
\end{figure}
%
%

\section{非直交系列の構成例}
図\ref{fig:example}は、本方式における非直交系列の構成の例である。
図\ref{fig:example}では。直交系列を3個連接することによって、非直交系列が構成
されている。もととなる直交系列と
$(+,+,+)$, $(+,+,-)$, $(+,-,+)$, $(+,-,-)$, $(-,+,+)$,
$(-,+,-)$, $(-,-,+)$, $(-,-,-)$
を乗算することで、構成される。
したがって、1つの直交系列から$8 (=2^3)$個の非直交系列が構成される。
連接数を$M_{con}$としたとき、1つの直交系列から$2^{M_{con}}$個の非直交系列が
構成される。直交系列長および直交系列数を$M_{os}$としたとき、
フレーム長は$M_{con} \times M_{os}$であり、非直交系列数$M_{non}$は
$M_{os} 2^{M_{con}}$である。
\begin{figure}[hpbt]
\begin{center}
  \includegraphics[width=0.8\textwidth]{seq-ex.eps}
  \caption{非直交系列の例}
\label{fig:example}
\end{center}
\end{figure}
%

\section{非直交系列の相互相関値の比較}
表\ref{tbl:correlation}に、
提案方式および M-ary 直交変調方式の相互相関値を示す。
表\ref{tbl:correlation}において、$M_{con}=3$であり、相関値は、
フレーム長で正規化する。本方式における相関値は
$\{1, \frac{1}{3}, 0, -\frac{1}{3}, -1 \}$であり、相関値が0となる系列数は、
$8(M_{os} -1)$である。M-ary/SS 型アロハ方式において、相関値は$\{1, 0\}$であり、
相関値が0となる系列数は$(M_{os} -1)$である。また、M系列を用いた
CSK 型アロハ方式において、相関値は$\{1, 0, -\frac{1}{3} \}$であり、
相関値が0となる系列数は$4(M_{os} -1)$である。相関値が0となる系列数は本方式が
もっとも多い。
\begin{center}
\begin{table}[hpbt]
 \caption{相互相関値}
 \label{tbl:correlation}
 \begin{center}
  \begin{tabular}{l|c|c} \hline
   \texttt{System} & Correlation value & Num. of seq.\\ \hline
   \texttt{提案方式} & $1$ & $1$ \\
   \texttt{$(M_{con}=3)$} & $\frac{1}{3}$ & $3$ \\
   \texttt{} & $0$ & $8(M_{os} -1)$ \\
   \texttt{} & $-\frac{1}{3}$ & $3$ \\
   \texttt{} & $-1$ & $1$ \\ \hline
   \texttt{M-ary 直交変調方式} & $1$ & $1$ \\
   \texttt{} & $0$ & $M_{os}-1$ \\ \hline
  \end{tabular}
 \end{center}
\end{table}
\end{center}


\chapter{性能解析}
本章では、非直交 CSK を用いた IM/DD 方式のシンボル誤り率および情報伝送効率の
の理論式を導出する。誤り率特性は1チップ時間にAPDの光入射面から吸収される
光子数をポアソン分布に従うものとし、APD 出力はガウス近似する。
また、シンチレーション、背景光、変調消光比、信号光によるショット雑音、
暗電流によるショット雑音として APD 表面漏れ電流、APD バルク漏れ電流、
電子回路の熱雑音を考慮し、フレーム同期が取れているものと仮定する。
本章で用いる記号を表\ref{tbl:notations2}に示す。

%
\begin{center}
\begin{table}[hpbt]
 \caption{本章で用いる記号}
 \label{tbl:notations2}
 \begin{center}
  \begin{tabular}{l|l} \hline
記号 & 記号の意味 \\ \hline
$T_c$ & チップ長 \\
$G$ & 平均APD利得 \\
$F$ & 過剰雑音指数 \\
$Me$ & 変調消光比 \\ 
$\eta$ & 量子効率 \\
$e$ & 電荷素量 \\
$\sigma_{th}^2$ & 熱雑音の分散 \\
$\sigma_s^2$ & シンチレーションの対数分散 \\
$P_b$ & 背景光電力 \\
$P_w$ & シンチレーションと背景光の影響を除いた受信光電力 \\
$h$ & プランク定数 \\
$\lambda_s$ & 光子の吸収率 \\
 & ($=\frac{\eta P_w}{hf}$) \\
$\lambda_b$ & 背景光による光子吸収率 \\
 & ($=\frac{\eta P_b}{hf}$) \\
$I_b$ & APDのバルク漏れ電流の平均値 \\
$I_s$ & APDの表面漏れ電流の平均値 \\ \hline
  \end{tabular}
 \end{center}
\end{table}
\end{center}

\section{シンボル誤り率}
1フレームを正しく判定する確率$P_c$は、次式のように表される。
\begin{align}
\label{Pc}
P_c&=\int_{0}^{\infty} P(X) \left[ 1 - \frac{1}{2} \mathrm{erfc} \left( \frac{\mu_0(X)}{\sqrt{2\sigma_0^2 M_{con}}} \right) \right] \times  \big(1-P_{os}(X)\big) dX,
\end{align}
ここで、$\mathrm{erfc}(x)$は誤差補関数であり、
$\mathrm{erfc}(x) = \frac{2}{\sqrt{\pi}} \int_x^{\infty} \exp(-t^2) dt$で
表される。
$P(X)$はシンチレーション$X$の確率密度関数であり、
\begin{align}
P(X)&=\frac{1}{\sqrt{2\pi \sigma_s^2}X} \exp\left[-\frac{\left(\ln X + \frac{\sigma_s^2}{2} \right)^2}{2\sigma_s^2} \right] \nonumber
\end{align}
で表される。対数分散$\sigma_s^2$は大気の状態や伝送距離により決定される。
$\mu_0(X), \sigma_0^2(X)$は、それぞれ、要素系列の極性に対する平均と分散であり、
\begin{align}
\mu_0&=GT_c \left( \frac{L_f}{2} \lambda_s X - \frac{L_f}{2}\frac{\lambda_s X}{Me} \right) \\
\sigma_0^2&=G^2 FT_c \left[ \frac{L_f+1}{2} \lambda_s X + \frac{L_f^2-1}{2} \frac{\lambda_s X}{Me} + L_f\lambda_b + \frac{2I_b}{e} \right] + \frac{2I_sT_c}{e} + 2\sigma_{th}^2 
\end{align}
で表される。
$P_{os}(X)$は、送信された直交系列の推定を誤る確率である。
直交系列$x_1$が送信されたとき、$P_{os}(X)$は、次式のように表される。
%
\begin{align}
P_{OS}(k)&=1 - \int_{-\infty}^{\infty} f(x_1) 
\left[ \int_{-\infty}^{x_1} f(x_{\stackrel{j}{j \neq 1}},X) dx_j \right]^{M_{os}-1} dx_1
\end{align}
%
ここで、 $f(x_j)$は直交系列$x_j$の相関器出力に関する確率密度関数であり、
\begin{align}
f(x_j,X)&=\underbrace{g(|q_j|,X) \otimes \cdots \otimes g(|q_j|,X)}_{M_{con}~times},  \\
g(q_j,X)&=\frac{1}{\sqrt{2 \pi \sigma_{j}^2(X)}} \exp \left[ -\frac{(q_j - \mu_j(X))^2}{\sqrt{2\sigma_{j}^2(X) M_{con}}} \right],
\end{align}
%
で表される。ここで、"$\otimes$"は畳み込み積分を表す。
$q_j$は$j$番目の相関器の出力であり、$u_j(X), \sigma_{j}^2(X)$はそれぞれ
$q_j$の平均、分散であり、
\begin{align}
\mu_1&=GT_c \left( \frac{L_f+1}{2} \lambda_s X + \frac{L_f-1}{2}\frac{\lambda_s X}{Me} + \lambda_b \right) \\
\mu_j&=GT_c \left( \frac{L_f \lambda_s X}{Me} + \lambda_b \right) \\
\sigma_0^2&=G^2 FT_c \left[ \frac{L_f+1}{2} \lambda_s X + \frac{L_f^2-1}{2} \frac{\lambda_s X}{Me} + L_f\lambda_b + \frac{2I_b}{e} \right] + \frac{2I_sT_c}{e} + 2\sigma_{th}^2 
\end{align}
で表される。

以上より、シンボル誤り率$P_e$は
\begin{align}
\label{Pe}
P_e&=1-P_c \nonumber \\
&=1-\int_{0}^{\infty} P(X) \left[ 1 - \frac{1}{2} \mathrm{erfc} \left( \frac{\mu_0(X)}{\sqrt{2\sigma_0^2 M_{con}}} \right) \right] \times  \big(1-P_{os}(X)\big) dX,
\end{align}
で表される。

\section{情報伝送効率}
1チップ時間あたりに受信に成功した情報量を情報伝送効率と定義する。
提案方式の1フレームあたりに送信できる情報量は$\log_2 M_{os} + M_{con}$ (bit)で
ある。したがって、
SIK 方式、M-ary 直交変調方式、提案方式のシンボル誤り率をそれぞれ
$P_{e_{SIK}}, P_{e_{M-ary}}, P_{e_{Pro}}$、符号長を$L_f$ (chip)、
CSKで用いる直交符号数を$M_{os}$とすると、
情報伝送効率$\eta_{D_{SIK}}, 
\eta_{D_{M-ary}}, \eta_{D_{Non}}$はそれぞれ、次式で表される。
\begin{align}
\eta_{D_{SIK}} &= \frac{(1-P_{e_{SIK}})}{L_f}, \nonumber \\
\eta_{D_{M-ary}} &= \frac{\log_2 M_{os} (1-P_{e_{M-ary}})}{L_f}, \nonumber \\
\eta_{D_{Non}} &= \frac{(\log_2 M_{os} + M_{con})(1-P_{e_{Pro}})}{L_f} \nonumber \\
\end{align}
%

\chapter{性能評価}
本章では、
SIK方式、直交CSK方式および提案方式のシンボル誤り率および情報伝送効率を
評価する。数値諸元を表\ref{tbl:parameters}に示す。
\begin{center}
\begin{table}[hpbt]
 \caption{数値諸元}
 \label{tbl:parameters}
 \begin{center}
  \begin{tabular}{l|l|l} \hline
名前 & 記号 & 値 \\ \hline
レーザ波長 & $f$ & 830 (nm) \\
チップ長 & $T_c$ & $4.0 \times 10^{-4}$ ($\mu$sec)\\
平均APD利得 & $G$ & 100 \\
背景光電力 & $ P_b$ & -45.0 (dBm) \\
APDのバルク漏れ電流の平均値 & $I_b$ & 0.1 (nA) \\
APDの表面漏れ電流の平均値 & $I_s$ & 10 (nA) \\
変調消光比 & $Me$ & 100 \\ 
受信機における雑音温度 & $T_r$ & 1100 (K) \\
負荷抵抗 & $R_L$ & 1030 ($\Omega$) \\
シンチレーションの対数分散 & $\sigma_s^2$ & 0.1 \\
電荷素量 & $e$ & $1.6 \time 10^{-19}$ (C)\\
量子効率 & $\eta$ & 0.6 \\
有効電離率 & $k_{eff}$ & 0.02 \\ \hline
  \end{tabular}
 \end{center}
\end{table}
\end{center}

図\ref{fig:graph-proposal-ser}に、
図\ref{fig:graph-cmp-ser}に、$M_{os}=32$、
1ビットあたりの受信光パワー$P_b=-57.0$ (dBm) としたときの
提案方式の連接数に対するシンボル誤り率を示す。
図より、連接数を増加するに従いシンボル誤り率特性が改善することがわかる。
これは、連接数の増加に伴いフレーム長が増加し、1シンボルあたりの
受信エネルギーが増加するためである。
%
\begin{figure}[hpbt]
\begin{center}
  \includegraphics[width=0.8\textwidth]{graph-pro-ser.eps}
\caption{提案方式の連接数に対するシンボル誤り率
($M_{os} = 8, P_{b}=-57.0$ (dBm))}
\label{fig:graph-proposal-ser}
\end{center}
\end{figure}
%

図\ref{fig:graph-cmp-ser}に、$M_{os}=32$、
1ビットあたりの受信光パワー$P_b=-57.0$ (dBm) としたときの
提案方式の連接数に対する情報伝送効率を示す。
情報伝送効率は連接数によって変化し、連接数の最適値が存在することがわかる。
直交系列数および直交系列長を固定し連接数を変化させた場合、\\
(1) フレーム長の増加によるビット誤り率特性の改善\\
(2) 使用する系列数の増加による情報ビット数の増加\\
(3) 1チップ時間あたりの情報ビット数の減少\\
(1)(2)と(3)の間にトレードオフが存在する。このことから、
提案方式には連接数の最適値が存在するものと考えられる。
連接雨の最適値についてはより詳しく検討する必要がある。
\begin{figure}[hpbt]
\begin{center}
  \includegraphics[width=0.8\textwidth]{graph-pro-efficiency.eps}
\caption{提案方式の連接数に対する情報伝送効率 
($M_{os} = 8, P_{b}=-57.0$ (dBm))}
\label{fig:graph-proposal-efficiency}
\end{center}
\end{figure}
%

図\ref{fig:graph-cmp-ser}に、$L_{f}=32$ (chip)としたときの
1ビットあたりの受信光パワーに対するシンボル誤り率を示す。
M-ary直交変調方式で用いる直交系列数を32、提案方式では、
$(M_{os}=8,~M_{con}=4),~(M_{os}=32,~M_{con}=1)$とする。
$(M_{os}=32~,M_{con}=1)$は陪直交変調方式と同じである。
図より、受信光パワーが高い時、M-ary直交変調方式が最も良シンボル誤り率特性を
示すことがわかる。
提案方式のシンボル誤り率はM-ary直交変調方式とSIK方式の中間であることがわかる。
提案方式は、符号間干渉が0ではない系列も使用しており、
その誤り率の影響を受けるため、M-ary直交変調よりもビット誤り率特性が低下する。
%
\begin{figure}[hpbt]
\begin{center}
  \includegraphics[width=0.8\textwidth]{graph-cmp-ser.eps}
\caption{SIK方式、M-ary直交変調方式、提案方式のシンボル誤り率
($L_{f} = 32$)}
\label{fig:graph-cmp-ser}
\end{center}
\end{figure}
%

図\ref{fig:graph-cmp-efficiency}に、$L_{f}=32$ (chip)としたときの
1ビットあたりの受信光パワーに対する情報伝送効率示す。
M-ary直交変調方式で用いる直交系列数を32、提案方式では、
$(M_{os}=8,~M_{con}=4),~(M_{os}=32,~M_{con}=1)$とする。
図より、受信光パワーにかかわらず、提案方式($M_{os}=8, M_{con}=4$)の情報伝送効率は3方式の中で最も
良いことがわかる。
このことから、情報伝送効率の関して、
符号間干渉の影響よりも
1フレームあたりに使用する系列数増加による効果のほうが高いことがわかる。
%
\begin{figure}[hpbt]
\begin{center}
  \includegraphics[width=0.8\textwidth]{graph-cmp-efficiency.eps}
\caption{SIK方式、M-ary直交変調方式、提案方式の情報伝送効率
($L_{f} = 32$)}
\label{fig:graph-cmp-efficiency}
\end{center}
\end{figure}
%

\chapter{結論}
本研究では、非直交 CSK を用いた IM/DD 方式を提案した。
特に、CSK で用いる非直交系列をシステマチックに構成する方法を提案した。
本研究では、シンチレーションが存在する環境下において、
シンボル誤り率および情報伝送効率を導出した。
また、従来の IM/DD 方式である、SIK 方式および 
M-ary 直交変調方式との比較を行った。その結果、以下のことを明かにした。
%
\begin{itemize}
 \item M-ary 直交変調方式のシンボル誤り率が最も良い。
 \item 提案方式は情報伝送効率を最大化する、連接数の最適値が存在する。
 \item 通信環境にかかわらず提案方式の情報伝送効率が最も良い。
\end{itemize}
%
以上より、非直交 CSK を用いた IM/DD 方式は光無線通信における
情報伝送効率向上法として有効な方式である。

今後の課題として、
提案方式における連接数の最適値に関する検討、
提案方式におけるシンボル誤り率低減法に関する検討、
情報伝送効率のさらなる向上法に関する検討、
情報変調の階層化に関する検討、
多元接続干渉を考慮した性能解析、などが挙げられる。

\begin{thebibliography}{99}
\bibitem{ocdma1} H.M. Kwon, "Optical Orthogonal Code-division Multiple-access 
system -- Part I: APD Noise and Thermal Noise," IEEE Trans. Communications,
vol.42, no.7, pp.2470--2479, July 1994. 
%
\bibitem{ocdma2} Y. Qiu, S. Chen, H.H. Chen, W. Meng, 
"Visible Light Communication based on CDMA Technology," IEEE Wireless Communications, 
vol.25, no.2, pp.178--185, Apr. 2018.
%
\bibitem{ocdma3} M.Y. Liu, T.L. Wang, S.M. Tseng, 
"Throughput Performance Analysis of Ascynchronous Optical CDMA Networks 
with Channel Load Sensing Protocol," IEEE Photonics Journal, 
vol.9, no.3, June 2017.
%
\bibitem{ocdma4} S.H. Chen, C.W. Chow, 
"Color-shift Keying Code-division Multiple-access Transmission for 
RGB-LED Visible Light Communications using Mobile Phone Camera," 
IEEE Photonics Journal, vol.7, no.6., Dec. 2014. 
%
\bibitem{ocdma5} K. Kiasaleh, 
"Performance of APD-based, PPM Free-space Optical Communication Systems in Atmospheric Turbulence," 
IEEE Trans. Communications, vol.53, no.9, pp.1455--1461, Sept. 2005. 
%
\bibitem{scinti} X. Zhu, J.M. Kahn, 
"Free-space Optical Comuunication through Atmospheric Turbulence Channels," 
IEEE Trans. Communications, vol.50, no.8, pp.1293--1300, Aug. 2002.
%
%
\bibitem{ocdma6} M. Hadi, M.R. Pakravan, 
"Analysis and Design of Adaptive OCDMA Passive Optical Networks," 
Journal of Lightwave Technology, vol.35, no.14, pp.2853--2863, 2017. 
%
\bibitem{ook} J. Li, M. Uysal,  
"Optical Wireless Communications: System Model, Capacity and Coding," 
Proc. IEEE VTC, vol.1, pp.168--172, Oct. 2003. 
%
\bibitem{sik1} T. Yamashita, M. Hanawa, Y. Tanaka, M. Takahara, 
"An Optical Code Division Multiplexing System using Hadamard Codes and SIK," 
IEICE Technical Report, OCS98-1, May 1998. 
%
\bibitem{sik2} Y. Kozawa, H. Habuchi, 
"Theoretical Ananlysis of Atomospheric Optical DS/SS with On-off Orthogonal 
M-sequence Pairs," Proc. ICICS, P0686, Dec. 2007. 
%
\bibitem{csk1} N. Ochiai, S. Kushibiki, T. Matsushima, Y. Teramachi, 
"Performance Analysis of Synchronous Optical CDMA System with EWO signaling," 
IEICE Trans. Fundamentals, vol.J86-A, no.9, pp957--968, Sept. 2003. 
%
%
\bibitem{csk} C-P. Hsieh, C-Y. Chang, G-C. Yang, W.C. Kwong, 
"A Bipolar-bipoloar Code for Asynchronous Wevelength-time Optical CDMA," 
IEEE Trans. Communications, vol.54, no.7, pp.2572--2578, July 2006. 
%
\bibitem{csk2} S. Takayanagi, H. Habuchi, Y. Kozawa, 
"Optical-wireless Enhanced Code-shift-keyging with IM/DD," 
Proc. APCC, 14-PM1-C, Oct. 2015.
\end{thebibliography}



%%% 謝辞
\acknowledgement
本研究を進めるするにあたり、貴重なご指導とご助言をくださいました
茨城大学工学部情報工学科教授 羽渕裕真先生、
茨城大学工学部情報工学科助教 小澤佑介先生に心より感謝いたします。
また、著者が研究を行う際にご協力を下さいました
千葉大学統合情報センター准教授 白木厚司先生、
千葉大学工学部情報画像学科小室研究室の学生諸氏、
千葉大学工学部情報画像学科白木研究室の学生諸氏、
茨城大学工学部情報工学科 羽渕研究室の学生諸氏、
研究を行うにあたり様々なご支援、ご協力を下さいました方々に
深く感謝の意を表します。
\begin{flushright}
  平成30年2月8日 小室 信喜
\end{flushright}

\end{document}

