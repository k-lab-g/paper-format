\chapter{序論}
近年、車車間通信や海中通信、ホームネットワークへの応用が期待できる
情報通信方式として、光無線通信が着目されている\cite{ocdma1}-\cite{csk2}。
光無線通信サービスの普及に向けたさまざまな標準化活動、
光無線通信の高度化および実用化に関する研究が行われており、
光強度変調/直接検波 (IM/DD: Intensity Modulation Direct Detection) が実用的な光無線通信方式として検討されている。
光強度変調方式では、
光変復調法\cite{ocdma4}-\cite{scinti}、
光無線通信路のモデル化\cite{ocdma5}、
疑似雑音系列を用いたIM/DD方式の情報変調法\cite{ook}-\cite{sik2}、
などが検討されている。

疑似雑音系列を用いたIM/DD方式における情報変調方式として、
オンオフキーイング (OOK: On-off Keying) 方式\cite{ook}や
シーケンスインバージョン (SIK: Sequence Inversion Keying) 方式\cite{sik1}\cite{sik2}が検討されている。
OOK方式は、送信データ"0"を送信する場合は疑似雑音系列を発生させず、送信データ"1"を
送信する場合は疑似雑音系列を発生させ送信する方式である。
OOK方式は、光強度の減衰が少ない
光ファイバ通信では実用的な方式であるが、光無線通信方式では、
シンチレーションが生じるため\cite{scinti}、
動的に理想的なしきい値を設定することが困難である。
SIK方式は直交する2つの疑似雑音系列を用い、情報に応じて送信符号を切り替える
方式である\cite{sik1}\cite{sik2}。
SIK方式は受信機でのしきい値判定を必要とせず、OOK方式と同じ
情報伝送効率を達成することができるため、光無線通信に適した方式である。
しかし1つの拡散系列あたり1ビットのデータを送信するため、
情報伝送効率の向上が課題である。

情報伝送効率の向上を図る情報変調として、M-ary 直交変調方式が
検討されている\cite{csk1}\cite{csk2}。
M-ary 直交変調方式は、符号シフトキーイング (Code Shift Keying; CSK) の
1種であり、伝送するデータに応じて$M$個の直交系列の中から1つの系列を選択し、
送信する方式である。
M-ary 直交変調方式は、直交系列数$M$を増加することに
よって、系列あたりの情報量を増加させることができる。
しかし、系列数を増加させるには系列長を長く必要があるため、情報伝送効率をまねく
おそれがある。

本研究では、光無線通信における情報伝送効率向上法として、非直交 CSK を
用いた光強度変調方式を提案する。特に、CSK で用いる非直交系列をシステマチックに
構成する方法を提案する。
CSK で用いる系列として非直交系列を用いることによって、
系列長を長くせずに情報変調に用いる系列数を増やすことができるため、
情報伝送効率の向上が期待できる。
シンチレーションが存在する環境下において、
提案方式のシンボル誤り率および情報伝送効率の理論式を導出し、
提案方式の有効性を示す。
