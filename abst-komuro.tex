\documentclass[a4j,10pt,twocolumn,oneside,notitlepage,fleqn,final]{jarticle}
%Preamble
\usepackage{fancyhdr} % ヘッダを変えるために必要
\usepackage{proc}
\usepackage{graphicx}
\usepackage[psamsfonts]{amssymb}
\usepackage{url}
\usepackage{color}
\usepackage{fancybox}
%\setlength{\topmargin}{-3.0cm} %上の余白を余計にとる
%\setlength{\oddsidemargin}{0.0cm} %左余白を余計にとる
%\setlength{\textheight}{25cm}
%\setlength{\mathindent}{0.1cm} %数式を左寄せにする
\renewcommand{\baselinestretch}{1.00} % 行間設定 1.0が通常、入らなかったら調整
%図と本文の調整、狭くしたかったら小さい値を設定
\setlength\textfloatsep{8pt}
%
% ヘッダの設定
%
\pagestyle{fancy} % ドキュメント全体
\renewcommand{\headrulewidth}{0.0pt} % ヘッダの下線を引かない。
%\lhead{\textgt{\textbf{平成30年度卒業論文概要 (千葉大学工学部情報画像学科 小室研究室)}}}
%\chead{\textgt{\textbf{平成30年度卒業論文概要 (千葉大学工学部情報画像学科 小室研究室)}}}
\lhead{}
\chead{}
\rhead{}
\lfoot{}
\cfoot{}
\rfoot{}
\begin{document}
\twocolumn[
\begin{center}
\vspace{-10mm} 
%\showtitle{\textbf{\Large{IoT・無線センサネットワークの基礎技術とその応用}}}{小室 信喜}{統合情報センター 学術情報部門}
\textbf{\Large{IoT・無線センサネットワークの基礎技術とその応用}} \\
\end{center}
%氏名と学籍番号を右寄せする場合
\begin{flushright}
小室 信喜 \\
統合情報センター・学術情報部門 %学籍番号
\end{flushright}
%\hrulefill \\ %タイトルと本文の間に罫線を引く
]
\section{はじめに}
研究室配属の時期が到来した。これまで学んだことを基盤として、
実際に研究に着手することになる。
研究すること\footnote{論文にまとめて発表するまで}は
フルマラソンに似ている\footnote{あくまで個人の見解である}。
辛い時もあるかもしれないが、研究室のメンバーで切磋琢磨しゴールを目指す。
ゴールした時には筆舌に尽くしがたい達成感がある。

当研究室は、真のユビキタス社会を実現すべく、試行錯誤し
情報通信の研究に勤しんでいる。
本稿では、当研究室の概要を紹介する。

\section{研究内容}
通信機能を備えたさまざまなモノを接続し、モノとモノ、
人とモノとの間で自律的に情報のやり取りを実現するというIoT
Internet of Things) およびその周辺技術の開発が加速している\cite{iot}。
我が国における科学技術政策としてSociety 5.0が提唱されており\cite{society5}、
国内外において、IoTや人工知能を核とした新たな技術分野の創出や市場の拡大、
それに伴う技術開発の重要性が高まっている。
%IoTは、環境・農業、防犯・防災、構造物センシングなどさまざまな分野での利用が
%検討されている\cite{iot}。


当研究室では、IoTを用いたシステムの開発・構築、
無線センサネットワークの高度化に関する研究を行っている。
主として、「IoTを用いた環境モニタリングシステムの開発」
「無線センサネットワークの設計」 について研究している。
IoTを用いた環境データモニタリングシステムの概略図を図\ref{fig:example1}に示す。
本研究では、環境データ取得センサの開発、
環境データ収集システムの構築、収集データの解析・環境状況の予測などを行う。
無線センサネットワークの設計に関する概略図を図\ref{fig:example2}に
示す。本研究では、IoTを用いたシステムからの
要求に応じた通信性能を提供する無線センサネットワークを設計・構築する。

\begin{figure}[th]
\includegraphics[width=0.48\textwidth]{data-collection.eps}
\caption{IoTを用いた環境データ収集システム}
\label{fig:data-collection}
\end{figure}

\begin{figure}[th]
\includegraphics[width=0.48\textwidth]{wsn.eps}
\caption{無線センサネットワークの設計}
\label{fig:wsn}
\end{figure}

\section{研究室について}
\begin{itemize}
\setlength{\itemsep}{-1mm}
\item 研究室の活動\\
\item 研究室の活動\\
平成30年度の構成は、教員が1名、B4が3名(2名が院進)である。
%方針は「研究室生活を楽しむ」である。
前期は輪講形式で勉強(通信の基礎、英文論文)し、後期以降本格的に研究に
打ち込む。研究成果は国際会議や論文を通じて対外発表する\footnote{ここが
ゴール}。
雰囲気は百聞不如一見、見学して確認されたい。
\item 研究室説明会\\
日時:1/10(木)、1/17(木)、1/24(木) 18時\footnote{白木研と合同で実施する}\\
場所:共用機器センター3階ゼミ室\footnote{アクセスは
石山研、白木研の資料を参照されたい}
\end{itemize}

\if0
\begin{figure}[bth]
\includegraphics[width=0.50\textwidth]{example1.eps}
\caption{IoTを用いた環境データ収集システム}
\label{fig:example1}
\end{figure}

\begin{figure}[tbh]
\includegraphics[width=0.50\textwidth]{example2.eps}
\caption{IoTネットワークの設計}
\label{fig:example2}
\end{figure}
\fi

\section{おわりに}
本稿では、小室研究室について紹介した。 研究室を決める際は研究内容だけではなく、
研究室の先生や先輩の雰囲気・印象、相性も考慮されたい。伝聞で判断せず、
色々な研究室を巡り、自分で見て判断することを勧める。
「当初は興味なかったが、見学してみたら自分に合ってそうだ」など
新たな発見があるかもしれない。

\small{
\begin{thebibliography}{9}%{}
\setlength{\itemsep}{-1mm}
\bibitem{iot} I. Yaqoob et al., 
"Internet of Things Architecture: Recent Advances, Taxonomy, Requirements, 
and Open Challenges," IEEE Wireless Communications, vol.24, no.3, pp.10--16, 
June 2017.
\bibitem{society5} "Society 5.0 -- 科学技術政策 --,"
\url{http://www8.cao.go.jp/cstp/society5_0/index.html} (平成30年11月30日閲覧)
\end{thebibliography}
}
\end{document}
