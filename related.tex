\chapter{関連研究}
\section{疑似雑音系列を用いた光強度変調システム}
光無線通信では、光強度変調/直接検波 (IM/DD: Intensity Modulation Direct Detection) が
実用的な方式として検討されている。
IM/DD 方式は、送信側で情報信号に応じて光の強度を変調し、
受信側で受信した信号の光強度を測ることにより、光子のエネルギーを検出し、
情報を復調する方式である。
近年では、光符号多分割多元接続 (OCDMA: Optical Code Division Multiple Access) 
への応用を想定し、疑似雑音系列を用いたIM/DD方式が検討されている\cite{ocdma1}-\cite{csk2}。

\subsection{疑似雑音系列を用いたIM/DD方式の原理}
送信機側では、レーザにより放射された光パルスが光スプリッタにより擬似雑音系列の
重み数$W$に分割される。次に疑似雑音符号の重み位置に対応した光時間遅延線を用いて、
分岐された$W$個の光パルスを時間遅延させ、最後に結合器によって光パルスを集め、
疑似雑音符号に従った長さ$L$ (chip)の光パルス列を送信する。
光無線通信路上では、送信信号は大気を伝搬する際の分子吸収による損失や、
粒子による散乱等の影響を大きく受ける。とくに、大気の屈折率変動によって
光ぱするの強度が変動してしまうシンチレーションという現象は光の強度に
情報をのせ送信する光強度変調方式に大きく影響を与える\cite{scinti}。
また、受信光パルス列が
レンズを通して受信機へ入射される際に、照明光や太陽光などの背景光雑音も
同時に入射してしまう。したがって、受信機では次の信号$r(t)$が入射される。
\begin{align}
r(t) &= \sqrt{P_w(t)} PN(t) X(t) + \sqrt{\frac{P_w(t)}{Me}}(1-PN(t)) X(t) + P_b(t)
\end{align}
ここで、$PN(t)$は疑似雑音系列、$P_w(t)$は疑似雑音系列の重みの位置の受信光パワー、
$\frac{P_w(t)}{Me}$は疑似雑音系列の重み位置以外の受信光パワー、
$X(t)$はシンチレーション、$P_b(t)$は背景光雑音、$Me$は変調消光比である。

受信機では、送信側で使用した系列と同一の系列を使用し、相関検波を行う。
相関検波では、受信信号はチップ間隔$T_c$毎にスイッチングされ、送信機とは逆の
時間遅延特性を持つ光遅延線によって、疑似雑音系列の重み位置の全光パルスが、
系列の最終チップに集められる。最終チップに集められた光パワー$P_{in}$は
系列長間隔$LT_c$におけるシンチレーションの影響を$X$とした場合、
\begin{align}
P_{in} &= W(P_w X + P_b)
\end{align}
で表される。

\subsection{疑似雑音系列を用いた光強度変調システムの情報変調方式}
疑似雑音符号を用いたIM/DD方式における情報変調方式として、
オンオフキーイング (OOK: On-off Keying) 方式\cite{ook}や
シーケンスインバージョン (SIK: Sequence Inversion Keying) 方式\cite{sik1}\cite{sik2}が検討されている。
\subsubsection{OOK方式}
OOK方式は光強度変調において最も単純な情報変調法である。OOK方式は、
送信データ"0"を送信する場合は疑似雑音系列を発生させず、送信データ"1"を
送信する場合は疑似雑音系列を発生させ送信する方式である。
受信機では、しきい値を設定することによってパルスの有無を判定し、
データを復調する。そのため、OOKではビット誤り率特性が最適となる理想的な
しきい値の設定が必要となる。したがって、光強度の減衰が少ない
光ファイバ通信では実用的な方式であるが、光無線通信方式では、
シンチレーションが生じるため、動的に理想的なしきい値を設定することが困難である。

\subsubsection{SIK方式}
SIK方式は直交する2つの疑似雑音系列を用い、情報に応じて送信符号を切り替える
方式である。SIK方式は受信機でのしきい値判定を必要とせず、OOK方式と同じ
情報伝送効率を達成することができる。送信機では、データに応じて
疑似雑音系列を1つ選択し、その疑似雑音系列に従って光パルスを時間拡散することで、
1ビットのデータを送信する。受信機では、送信側と同じ系列を用意し、
相関値が最大となる系列を送信系列と推定し、データを復調する。
したがって、受信機側でしきい値を必要とせず、光無線通信に適した方式である。
しかし1つの拡散系列あたり1ビットのデータを送信するため、
情報伝送効率の向上が課題である。
%

\section{M-ary 直交方式の原理}
M-ary 直交変調方式は符号シフトキーイング (CSK: Code Shift Keying) 方式\cite{csk1}\cite{csk2}の1種
である。CSKは、情報データに応じて拡散系列を変化させ、伝送する方式である。
SIK方式は1つの疑似雑音系列を用いて情報を伝送する方式であるのに対し、
CSK は疑似雑音系列の集合の中から情報データによって1つの系列を選択し、
その選択した系列によってデータを伝送する方式である。
M-ary 直交変調方式は$M$個の直交系列の集合の中から、
情報データによって1つの直交系列を選択し、
その選択した系列によって情報を伝送する方式である。
そのため1つの拡散系列あたりのビット数を増やすことができ、
情報伝送効率の向上が期待できる方式である。
%
\begin{figure}[hpbt]
\begin{center}
  \includegraphics[width=1.0\textwidth]{csk_system.eps}
\caption{M-ary 直交変調方式のシステム構成}
\label{csk_system}
\end{center}
\end{figure}
%
図\ref{csk_system}にM-ary 直交変調方式の
送信機および受信機の構成を示す。送受信機で利用する直交系列の集合を
$\{os_i(t); i=1, 2, \cdots, M\}$で表すものとする。送信側ではまず、
これらの$M$個の直交系列の中から$\log_2 M$ビットの情報によって1つの直交系列を
択する。次に、搬送波を乗算することによって、その系列を送信する。

受信側では送信側で用いる直交系列と同じ系列を用いる。
受信信号をチップ間隔ごとにAvalanche Photo Diode (APD)を用いて光電変換する。
光電変換した電気信号は、
要素系列毎に各直交系列$os_{j}~~(j=1,2, \cdots, M)$との相関を取る。
各直交系列の相関値の絶対値の和を求め、その値が最も大きくなる系列を、
送信された系列の要素系列であると判定し、情報を復調する。

%
