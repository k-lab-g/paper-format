\chapter{提案システムの構成}

\section{非直交系列構成法}
図\ref{fig:structure}に非直交系列の構成を示す。
本方式では、直交系列を$M_{con}$個連接することにより、非直交系列を構成する。
直交系列を$M_{con}$個連接することにより構成された系列を
1フレームとする。直交系列 \#$i$から構成される非直交系列のグループを
グループ\#$i$とする。提案方式において構成される系列は、
異なるグループの系列とは直交しているが、
同じグループの系列は非直交である。1つの直交系列から $2^{M_{con}}$個の
非直交系列が構成される。したがって、$M_{os}$個の直交系列からは、
$2^{M_{con}} M_{os}(=M_{non})$個の非直交系列が構成される。また、
1非直交系列当たりの情報ビット数は
$N_{bit} = \log_2 M_{os} + M_{con}$[bit] である。
直交系列を長 $L_{os}$ とすると、フレーム長$L_f$は $M_{con} \times L_{os}$ となる。
%
%
\begin{figure}[hpbt]
\begin{center}
  \includegraphics[width=0.75\textwidth]{sequence.eps}
\caption{非直交系列の構成}
\label{fig:structure}
\end{center}
\end{figure}
%

\section{送受信機}
図\ref{fig:model}に本方式の送受信機の構成を示す。
送信側では、まず、送信データ ($N_{bit}$(bit) の情報)を
$\log_2 M_{os}$ (bit)と$M_{con}$ (bit)に分割する。
$\log_2 M_{os}$ (bit)のデータに基づいて$M_{os}$個の直交系列の中から
1つの系列を選択する。次に、$M_{con}$ (bit)の極性に応じて、選択した直交系列を連接する。
最後に、構成された連接系列を送信する。
%

受信側では送信側で用いる直交系列と同じ系列を用いる。
受信信号をチップ間隔ごとにAvalanche Photo Diode (APD)を用いて光電変換する。光電変換した電気信号は、
要素系列毎に各直交系列$OS_{j}~~(j=1,2, \cdots, M_{os})$との相関を取る。
各直交系列の相関値の絶対値の和を求め、その値が最も大きくなる系列を、
送信された系列の要素系列であると判断する。また、相関器出力の正負から
連接系列の極性を判定し、データを復調する。
\begin{figure}[hpbt]
\begin{center}
  \includegraphics[width=1.0\textwidth]{system.eps}
\caption{提案方式のシステムモデル}
\label{fig:model}
\end{center}
\end{figure}
%
%

\section{非直交系列の構成例}
図\ref{fig:example}は、本方式における非直交系列の構成の例である。
図\ref{fig:example}では。直交系列を3個連接することによって、非直交系列が構成
されている。もととなる直交系列と
$(+,+,+)$, $(+,+,-)$, $(+,-,+)$, $(+,-,-)$, $(-,+,+)$,
$(-,+,-)$, $(-,-,+)$, $(-,-,-)$
を乗算することで、構成される。
したがって、1つの直交系列から$8 (=2^3)$個の非直交系列が構成される。
連接数を$M_{con}$としたとき、1つの直交系列から$2^{M_{con}}$個の非直交系列が
構成される。直交系列長および直交系列数を$M_{os}$としたとき、
フレーム長は$M_{con} \times M_{os}$であり、非直交系列数$M_{non}$は
$M_{os} 2^{M_{con}}$である。
\begin{figure}[hpbt]
\begin{center}
  \includegraphics[width=0.8\textwidth]{seq-ex.eps}
  \caption{非直交系列の例}
\label{fig:example}
\end{center}
\end{figure}
%

\section{非直交系列の相互相関値の比較}
表\ref{tbl:correlation}に、
提案方式および M-ary 直交変調方式の相互相関値を示す。
表\ref{tbl:correlation}において、$M_{con}=3$であり、相関値は、
フレーム長で正規化する。本方式における相関値は
$\{1, \frac{1}{3}, 0, -\frac{1}{3}, -1 \}$であり、相関値が0となる系列数は、
$8(M_{os} -1)$である。M-ary/SS 型アロハ方式において、相関値は$\{1, 0\}$であり、
相関値が0となる系列数は$(M_{os} -1)$である。また、M系列を用いた
CSK 型アロハ方式において、相関値は$\{1, 0, -\frac{1}{3} \}$であり、
相関値が0となる系列数は$4(M_{os} -1)$である。相関値が0となる系列数は本方式が
もっとも多い。
\begin{center}
\begin{table}[hpbt]
 \caption{相互相関値}
 \label{tbl:correlation}
 \begin{center}
  \begin{tabular}{l|c|c} \hline
   \texttt{System} & Correlation value & Num. of seq.\\ \hline
   \texttt{提案方式} & $1$ & $1$ \\
   \texttt{$(M_{con}=3)$} & $\frac{1}{3}$ & $3$ \\
   \texttt{} & $0$ & $8(M_{os} -1)$ \\
   \texttt{} & $-\frac{1}{3}$ & $3$ \\
   \texttt{} & $-1$ & $1$ \\ \hline
   \texttt{M-ary 直交変調方式} & $1$ & $1$ \\
   \texttt{} & $0$ & $M_{os}-1$ \\ \hline
  \end{tabular}
 \end{center}
\end{table}
\end{center}

