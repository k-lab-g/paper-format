\documentclass[a4j,10pt]{jarticle}
\usepackage{graphicx}
\usepackage{ascmac}
\setlength{\textwidth}{160mm}
\setlength{\textheight}{246mm}
\setlength{\oddsidemargin}{-0.4mm}
\setlength{\evensidemargin}{-0.4mm}
\setlength{\topmargin}{-5.4mm}
\setlength{\headheight}{0mm}
\setlength{\headsep}{0mm}
\setlength{\footskip}{5mm}
\begin{document}
\begin{center}
\large
研究メモ
\end{center}
\begin{flushright}
小室 信喜 \\
千葉大学統合情報センター
\end{flushright}
\hrulefill
\section{A: 研究テーマは何ですか?} 
研究の主題を書く
%
%
\section{B: なぜその研究テーマを選んだのですか?}
\subsection{B1: 背景} 
その研究テーマを選んだ理由を書く。
%
%
\subsection{B2: 従来の研究内容} 
その研究テーマに関連してこれまでに行われた研究を書く
%
%
\subsection{B3: 従来での未解明点} 
その研究テーマに関連してこれまでに行われた研究における未解明点を書く
%
%
\subsection{B4: 目的} 
その研究テーマを選んだ目的を書く
%
%
\section{C: どのような手段を使ってその研究を行いましたか?}
\subsection{C1: 使用した手段(簡潔)} 
その目的を実現するために使用した手段を簡潔に書く
%
%
\subsection{C2: 使用した手段(詳細)} 
その目的を実現するために使用した手段を詳細に書く
%
%
\section{D: どのようなデータがその研究結果として得られましたか?}
\subsection{D1: 得られた事実(簡潔)} 
その使用した手段により得られた事実を簡潔に書く
%
%
\subsection{D2: 得られた事実(詳細)} 
その使用した手段により得られた事実を詳細に書く
%
%
\section{E: どのように従来及び今回の研究結果が対比されましたか?} 
従来の研究結果と今回得られた結果を比較する
%
%
\section{F: どのような結論が得られましたか?}
\subsection{F1: 得られた結論(簡潔)} 
データを論理的に分析して得られた結論を簡潔に書く
%
%
\subsection{F2: 得られた結論(詳細)} 
データを論理的に分析して得られた結論を詳細に書く
%
%
\section{G: 今回の研究から浮かびあがった今後の課題がありましたか?} 
浮かび上がった問題点を今後の課題として提案する。
%
\end{document}
